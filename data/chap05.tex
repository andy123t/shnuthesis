
% 表格环境

\chapter{表格环境}

\section{表的使用}

LaTeX 的 table 环境是一个浮动体, 与 figure 环境的排版方式类似. 简单的表格可以使用三线表进行排版, 其结构是仅在标题的上下以及表格的最下方添加横线.

排版表格最基本的环境是 tabular 环境, 适合大多数需求. 如果想要控制表格行列间距, 可使用命令 \verb|\tabcolsep| 和 \verb|\arraystretch|, 它们分别控制列间距和行间距. 例如, \verb|\tabcolsep{10pt}| 表示列间距为 10pt, 默认是 6pt,  \verb|\arraystretch{1.2}| 表示行间距为 1.2 倍, 默认是 1 倍间距.

本文基于 array 宏包定义了新的列类型 \verb|LCR| 格式, 实现了居左、居中和居右的对齐方式, 可用于 tabularx 环境. tabularx 环境要求先指定表格的总宽度, \verb|LCR| 三个格式根据表格的总宽度自动调整列宽, 使其相等. 本模板还定义了新的列类型 \verb|P{}|, 可用于 tabular 和 tabularx 环境, 它允许指定某列的宽度并使其内容居中 (如 \verb|P{1cm}| 控制某一列的宽度为 1cm), 实际上 \verb|P{}| 是在 \verb|p{}| 的基础上增加了居中功能.


\section{表格示例}

使用 tabular 环境, 如下表格: 表~\ref{tab:foo}. 通过 \verb|autoref| 引用表格: \autoref{tab:foo}.

\begin{table}[htp!]
\centering
\setlength{\tabcolsep}{12pt}  % 6pt standard
\renewcommand{\arraystretch}{1.2}
\caption{学术活动安排样例}
\label{tab:foo}
\begin{tabular}{|c|c|c|c|}
\hline
\textbf{日期}  & \textbf{地点} & \textbf{活动名称} & \textbf{备注} \\ \hline
2024年8月1日      & 上海       & 学术研讨会      & 主题:人工智能 \\ \hline
2024年8月15日    & 北京       & 学术交流会      & 重点:数学建模 \\ \hline
2024年9月1日      & 深圳       & 研究研讨会      & 主题:数据科学 \\ \hline
2024年10月15日  & 广州       & 创新论坛         & 重点:科技创新 \\ \hline
\end{tabular}
\end{table}

\clearpage

\begin{table}[htp!]
\centering
\caption{论文进度安排}
%\renewcommand\arraystretch{1.1}
\begin{tabular}{|P{4cm}|P{7cm}|}
\Xhline{2\arrayrulewidth}
论文起止时间    &  论文筹备过程 \\
\hline
20xx.xx -- 20xx.xx  &  论文定题, 整理相关文献 \\
\hline
20xx.xx -- 20xx.xx  &  审查、修改、完成开题报告 \\
\hline
20xx.xx -- 20xx.xx  &  对论文排版、初步完成论文初稿 \\
\hline
20xx.xx -- 20xx.xx  &  毕业论文预答辩 \\
\hline
20xx.xx -- 20xx.xx  &  对论文进行补充、完善 \\
\hline
20xx.xx -- 20xx.xx  &  论文定稿 \\
\hline
20xx.xx -- 20xx.xx  &  毕业论文答辩 \\
\Xhline{2\arrayrulewidth}
\end{tabular}
\end{table}


使用 tabularx 环境, 如下表格: 表~\ref{tab:heightweight}.

\begin{table}[htp!]
\centering
% PLCR已经定义
\caption{某校学生身高体重样本}
\renewcommand\arraystretch{0.92}
\label{tab:heightweight}
\begin{tabularx}{0.9\textwidth}{P{1.5cm}CCC}
\toprule
序号 & 年龄 & 身高 & 体重 \\
\midrule
001 & 15 & 156 & 42 \\
002 & 16 & 158 & 45 \\
003 & 14 & 162 & 48 \\
004 & 15 & 163 & 50 \\
\cmidrule{2-4}
平均 & 15 & 159.75 & 46.25 \\
\bottomrule
\end{tabularx}
\end{table}


基于 tabular 环境设置一些格式: 上下表格线加粗, 如表~\ref{tab:error}.

\begin{table}[htp!]
%\small
\centering
%\setlength{\tabcolsep}{8pt}
%\renewcommand\arraystretch{1.1}
\caption{数值误差与收敛速率示例}
\label{tab:error}
\begin{tabular}{c|c|cc|cc|cc}
\Xhline{2\arrayrulewidth}
逼近次数 & 步长 $h$ & $L^2$ 误差 & 收敛阶 & $H^1$ 误差 & 收敛阶 & $L^\infty$ 误差 & 收敛阶 \\
\hline
  & 1/128 & 9.18E-06 & 2.02 & 7.70E-03 & 1.01  & 6.46E-07 & 2.02 \\
1 & 1/256 & 2.29E-06 & 2.01 & 1.92E-03 & 1.00  & 1.61E-07 & 2.01 \\
  & 1/512 & 5.70E-07 & 2.00 & 9.56E-04 & 1.00  & 4.01E-08 & 2.00 \\
\hline
  & 1/128 & 1.39E-08 & 3.01 & 1.15E-05 & 2.01  & 3.48E-12 & 4.02 \\
2 & 1/256 & 1.73E-09 & 3.01 & 2.88E-06 & 2.01  & 3.27E-13 & 3.94 \\
  & 1/512 & 2.17E-10 & 3.00 & 7.24E-06 & 2.00  & 6.66E-13 & 1.55 \\
\hline
  & 1/32  & 2.28E-09 & 4.05 & 6.92E-07 & 3.04  & 1.45E-15 & 8.21 \\
3 & 1/64  & 1.42E-10 & 4.03 & 8.65E-08 & 3.02  & 2.06E-14 & 3.85 \\
  & 1/128 & 8.91E-12 & 4.01 & 1.08E-08 & 3.01  & 3.86E-14 & 0.91 \\
\Xhline{2\arrayrulewidth}
\end{tabular}
\end{table}

