
% 引言

\chapter{引言}\label{chap:Intro}

\section{研究背景}\label{sec:background}

这是小四号的正文字体, 行间距 1.35 倍.

通过空一行实现段落换行, 仅仅是回车并不会产生新的段落.

自定义一个命令 \verb|\red{文字}| 来\red{加红文字}, 在论文修改阶段方便标记.

本模板 \href{https://github.com/andy123t/shnuthesis}{\texttt{shnuthesis}} 基于标准文类 ctexbook 设计, 可以在目前主流的 \href{https://en.wikibooks.org/wiki/LaTeX/Introduction}{\LaTeX{}} 编译系统中使用, 如 \TeX{}Live 和 MiK\TeX{}. 因 C\TeX{} 套装已停止维护, \textbf{不再建议使用}.


\section{主要结论}\label{sec:mainResults}

本模板定义了以下选项:
\begin{itemize}
  \item \verb|master| \quad 硕士学位论文, 默认可省略
  \item \verb|doctor| \quad 博士学位论文, 不能省略
  \item \verb|arts| \quad 文科学位论文, 默认缺省为理科
  \item \verb|print| \quad 用于打印, 封面等生成空白页
\end{itemize}

\textbf{注:} 提交给图书馆的论文电子版不要选 print.


\section{结构安排}

本文接下来的写作安排如下:

第二章, 我们介绍了 LaTeX 常用环境, 包括列表的使用、文献引用、数学公式、定理环境以及算法环境.

第三章, 对于差分方法数值求解微分方程, 给出了一个简短的示例.

第四章, 针对插图环境, 给出了单个图形居中放置、两个图形并排放置以及多个图形并排放置的示例.

第五章, 针对表格环境, 介绍了一些自定义命令, 并给出相应的表格示例.

最后是参考文献、附录、致谢和攻读硕士学位期间的研究成果.

