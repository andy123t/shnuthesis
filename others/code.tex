% !Tex Program = xelatex
% matlab and python code highlighting

\documentclass[a4paper,zihao=-4]{ctexart}

\usepackage{geometry}
\geometry{left=3.0cm,right=3.0cm,top=2.9cm,bottom=3.2cm}
\linespread{1.35}

% package for code highlighting
\usepackage{codestyle}

\usepackage{fontspec}
\pagenumbering{gobble}

\title{Code highlighting}
\author{Author}
\date{}


\begin{document}

%\maketitle


%\section{代码环境}

这是一个代码环境.

\begin{lstlisting}[language=TeX,title={Title of source code}]
\documentclass{article}  % book,report,letter,beamer
\usepackage{amsmath}  % AMS package
\begin{document}
Hello world!
$$E = m c^2$$
\end{document}
\end{lstlisting}


% 已定义  style=matlab 以及 style=python

这是 MATLAB 程序代码高亮环境.

\begin{lstlisting}[style=Matlab,basicstyle=\footnotesize\fontspec{Courier New},title={MATLAB code}]
% Euler method for the ODEs
clear all;  clf
h=0.1; x=0:h:1;
N=length(x)-1;
u(1)=0;
fun=@(t,u) t.^2+t-u;
for n=1:N
    u(n+1)=u(n)+h.*fun(x(n),u(n));
end
ue=-exp(-x)+x.^2-x+1;
plot(x,ue,'b-',x,u,'r+','LineWidth',1)
xlabel('x'), ylabel('u')
\end{lstlisting}


这是 Python 程序代码高亮环境.

\begin{lstlisting}[style=python,basicstyle=\footnotesize\fontspec{Consolas},title={Python code}]
# Fibonacci series up to n
def fib(n):
    a, b = 0, 1
    while a < n:
        print(a, end=' ')
        a, b = b, a+b
    print()
fib(1000)
\end{lstlisting}



\end{document}

